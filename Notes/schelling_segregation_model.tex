\documentclass{article}

\begin{document}
\section{Introduction}

Schelling's model of segregation is an agent-based model developed by economist Thomas Schelling in 1971. This model demonstrates how even mild preferences for neighbors of the same type (e.g., race, ethnicity, income level) can lead to high levels of segregation at the societal level. The model consists of the following key elements:

\begin{itemize}
\item Two types of agents representing different groups, initially randomly distributed on a grid.
\item Agents are satisfied if at least a certain fraction (the "tolerance threshold") of their neighbors are of the same type.
\item In each round, unsatisfied agents move to the nearest vacant spot where they would be satisfied.
\item The simulation continues until all agents are satisfied or a stable configuration is reached.
\end{itemize}

In his work, Schelling used two types of coins as the agents of the model. Since I have a background in physics and there is often a silly and somewhat funny rivalry between physicists and engineers, I will use the neighborhoods of physicists and engineers in this model instead of those coins.

At the end, I will also explore some methods of preventing segregation, such as variable thresholds or introducing third stabilizer agents that appear to reduce segregation.

\end{document}
